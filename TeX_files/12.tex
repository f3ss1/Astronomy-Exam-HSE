\section{Источники энергии нейтронных звезд}

Выделяют 4 основных источника энергии нейтронных звёзд:

\begin{enumerate}
	\item \textbf{Вращение}
	
	Самый эффективный способ для того, чтобы запасти много энергии. Замедляется вращение звезды, из-за чего энергия переходит в излучение: быстрое вращение (радиопульсары имеют периоды собственного осевого вращения примерно от 1 мс до 10 с) Нейтронные звёзды с сильным магнитным полем приводит к генерации электромагнитного из-лучения и потока релятивистских частиц. Кроме импульсов в радио-диапазоне, у ряда источников также зарегистрированы импульсы в других диапазонах спектра.
	
	\item \textbf{Аккреция}
	
	Сначала нейтронные звёзды были открыты как аккрецирующие рентгеновские источники в двойных системах. В двойной системе вещество с одной звезды может попадать на второй объект (например, обычную звезду) двумя способами:
	
	\begin{itemize}
		\item Звездный ветер
		
		Его часть может быть гравитационно захвачено вторым объектом, вследствие вещество будет выпадать на поверхность. При падении на поверхность нейтронной звезды энерговыделение очень большое.
		
		\item Больший темп перетекания.
		
		Вокруг каждой звезды в двойной системе существует область, называемая полостью Роша. Внутри этой области вещество гравитационно связано со звездой, но если поместить вещество в эту область без дополнительной скорости, то оно уже не будет гравитационно связано с этой звездой и будет теряться. Если звезда заполнит свою полость Роша, то вещество будет вынуждено перетекать внутрь полости Роша второго объекта. Звезда может заполнить полость Роша или в результате расширения (превратилась в красный гигант), или в результате уменьшения большой полуоси системы (по каким-либо причинам). 
	\end{itemize}

	Аккреция есть самый простой, но при этом самый мощный источник энергии в мире из тех, что могут давать большой выход энергии. При падении вещества на нейтронную звезду выделяется до 10\% от $mc^2$.
	
	\item \textbf{Магнитные поля}
	
	Нейтронные звёзды, чья активность в основном связана с выделением энергии магнитного поля, называют магнитарами. Магнитные поля порождаются электрическими токами, и энергию этих токов можно излучать. Обычно энергии поля очень велики.
	Данное поле сильно эволюционирует. Связано это с тем, что в коре текут токи, которые порождают магнитное поле. Магнитары излучают энергию токов, которые текут в коре звезды.
	
	Энергию токов можно выделять двумя способами:
	
	\begin{itemize}
		\item Медленно, например, как в случае нагревания ноутбука.
		
		\item Быстро в виде вспышек. Не многие «запасы энергии» можно расходовать так быстро, в этом особенность вспышек. Максимальная наблюдавшаяся светимость магнитара $10^47$ эрг/с, это ярче всей галактики!
	\end{itemize}

	\item \textbf{Тепловая энергия}
	
	Рождаясь очень горячими, нейтронные звезды остывают со временем в начале за счет излучения нейтрино, а затем --- за счет излучения фотонов с поверхности.
	
	\begin{itemize}
		\item Нейтринное охлаждение
		
		Это процесс охлаждения звёздных недр образующимися в них нейтрино, которые свободно уносят энергию из всего объёма ядра, так как звезда прозрачна для нейтрино низких энергий. Скорость такого объёмного нейтринного охлаждения, в отличие от классического поверхностного фотонного охлаждения, не лимитирована процессами переноса энергии из недр звезды к её фотосфере, поэтому такой механизм охлаждения весьма эффективен.
		
		\item Остывание, связанное с фотонами, когда звезда становится холодной и нейтрино излучается не так много. Поэтому звезда излучает с поверхности.
	\end{itemize}
\end{enumerate}